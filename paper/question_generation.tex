%
% File acl2016.tex
%
%% Based on the style files for ACL-2015, with some improvements
%%  taken from the NAACL-2016 style
%% Based on the style files for ACL-2014, which were, in turn,
%% Based on the style files for ACL-2013, which were, in turn,
%% Based on the style files for ACL-2012, which were, in turn,
%% based on the style files for ACL-2011, which were, in turn, 
%% based on the style files for ACL-2010, which were, in turn, 
%% based on the style files for ACL-IJCNLP-2009, which were, in turn,
%% based on the style files for EACL-2009 and IJCNLP-2008...

%% Based on the style files for EACL 2006 by 
%%e.agirre@ehu.es or Sergi.Balari@uab.es
%% and that of ACL 08 by Joakim Nivre and Noah Smith

\documentclass[11pt]{article}
\usepackage{acl2016}
\usepackage{times}
\usepackage{url}
\usepackage{latexsym}
\usepackage{graphicx}

%\aclfinalcopy % Uncomment this line for the final submission
%\def\aclpaperid{***} %  Enter the acl Paper ID here

%\setlength\titlebox{5cm}
% You can expand the titlebox if you need extra space
% to show all the authors. Please do not make the titlebox
% smaller than 5cm (the original size); we will check this
% in the camera-ready version and ask you to change it back.

\newcommand\BibTeX{B{\sc ib}\TeX}

\title{Are you asking the right questions? \\ Question generation for filling missing information}

\author{First Author \\
  Affiliation / Address line 1 \\
  Affiliation / Address line 2 \\
  Affiliation / Address line 3 \\
  {\tt email@domain} \\\And
  Second Author \\
  Affiliation / Address line 1 \\
  Affiliation / Address line 2 \\
  Affiliation / Address line 3 \\
  {\tt email@domain} \\}

\date{}

\begin{document}
\maketitle
\begin{abstract}

\end{abstract}
\section{Introduction}

Asking question is an important aspect of communication and understanding. In the field of Neurolinguistics Programming, \cite{o2011introducing} go so far as saying -- ``The only answer to the question `What does a word really mean?' is `To whom?'. The way we know we understand someone is by giving their words meaning. Our meaning. Not their meaning. And there is no guarantee that the two meanings are the same." The primary reason for this being each person has a mental map of their own. As a way to circumvent this issue, Neurolinguistics Programming introduces Meta Model \cite{bandler1975structure} which is a series of questions that seek to reverse and unravel the deletions and distortions and generalizations of language. These questions aim to fill in the missing information, reshape the structure and elicit specific information to make sense of the communication.

In the field of natural language processing, there is an understanding of the importance of asking questions. However most previous work has been focused on generating questions whose answers can be found in some aforementioned text. If we wish to make computers better at understanding a given text, we should also enable them to ask questions about the missing information in the text; a text that was created using a different mental map in mind. In our work, we for the first time introduce a model that can learn to ask questions about missing information in a text. To train our model we use an idea from decision theory called the Value of Perfect Information (VPI) which measures the worth of collecting additional information. Using this idea we make the decision of whether to ask a question or not and asking which question would increase the expected utility of a given text; where the expectation is over all the possible answers to the question. 

Our second contribution is the introduction of a dataset that can be used to train our question generation model. We use the data dump from stackexchange.com which is a network of question-and-answer websites on topics in varied fields. Typically, a user posts a question or a concern and other users answer them. However, a lot of these posts remain unanswered. \cite{asaduzzaman2013answering} show that one of the main reasons for posts remaining unanswered is that they are not clear enough or they are missing some information. Figure 1 shows an example of an unanswered post. These unanswered posts rightly fit our use case of a text with potential missing information that can be filled by asking the right question. We make the observation that in several instances, users post comments to a given post asking either for some clarification or for some missing information. Subsequently, the author of the post updates the post adding that piece of missing information. Figure 1 shows an example of such an update. We make use of this observation to create our dataset of \{\textit{post, question, answer}\} triples; where the \textit{post} is the initial post; the \textit{question} is the question comment on the post, and the \textit{answer} is the update made to the post. Our task is defined as given an unanswered post, decide if the post is complete or incomplete (has something missing) and in case of an incomplete post, ask a question. 

\iffalse
- The only answer to the question 'What does a word really mean?' is 'To whom?'
- Words can be ambiguous and so humans learn to ask questions explicitly when they have not understood something \\
- To quote NLP from Psychology \cite{o2011introducing} 'How do we know we understand someone? By giving their words meaning. Our meaning. Not  their meaning. And there is no guarantee that the two meanings are the same.' \\
- The NLP Meta Model \cite{bandler1975structure} is a series of questions that seek to reverse and unravel the deletions and distortions and generalizations of language. These questions aim to fill in the missing information, reshape the structure and elicit specific information to make sense of the communication.
- Meta Model says that each person has a mental map of his own
- Given a description meta model is a systematic way of asking a series of questions that would help gather more information 
- Can we teach computers to ask questions about something that is not clear \\
- If we want to make computers better at understanding a given information, we should enable it to ask questions with respect to its mental map

- Asking the right questions is the key to understanding any content better \\
- Asking intelligent questions is hard. We as humans are poor at it too \\
- In education, there has been work in understanding the importance of asking questions for improving the process of learning \\
- Most previous work in question generation has been reading comprehension type question where the answer is found in the text \\
- However the purpose that we ask questions in our day to day life -- not to test someone's understanding about a text, but to clarify the content of the text or to point out missing information in the text \\
- Although there is a understanding that asking such questions is important (cite work from other fields), there has not been any work in the computational world before \\
- We, for the first time, introduce a novel method to generate questions that help clarify the content \\
- We use data from stackexchange.com where users post issues and other users post solutions to it. However a lot of posts go unanswered because they are not clear enough. \\
- We model our task as whether we can ask such clarifying questions to the post.  We pick the question that maximizes the expected utility of the post
- We take inspiration from value of perfect information in AI and introduce a neural network model 

\section{Related Work} \label{related_work}

'Deep Questions without Deep Understanding' 
- generate high-level (they call it deep) question templates by crowdsourcing and then given a text segment, rank question templates that are relevant. 
- We on the other hand make use of existing resources to generate question templates. 

Automatic Factual Question Generation from Text 
- Normal RC type questions generation task
- This is a thesis, so related work section is exhaustive and so can be useful

Automatic Question Generation for Literature Review Writing Support
- Generates questions that can help author write the lit review better
- Input to system is a literature review and o/p is a set of questions
- The system captures all the citations in the paper, extract features out of it and then uses fixed templates to ask questions about the content
- The introduction/motivation of paper is good
						
Generating Natural Questions About an Image 
- Task: Visual Question Generation -- Different from image captioning since the question asked who help infer something from the image that is not directly shown in the image
- Contributions - created new dataset, collected gold annotations and showed that image captioning doesn't do well at this, analyzed several methods and showed deep learning models outperform, created evaluation metric delta-BLEU for this task
			
The Importance of Being Important: Question Generation 
- Why is asking intelligent questions important
- Some related work are relevant

Question Generation from Concept Maps
- Research question - how are questions generated in tutoring systems
- Domain - dialogue between tutor and a student
- They have drawn connections with work in learning/psychology
- Concept map is nothing but a graph with nodes (key terms) and edges (relations like is-a, has-a, etc)
- Their method is to generate concept maps and then have rules to generate questions using these maps
- Highly cited work

Meta Model in Intro to Neuro-Linguistic Programming
- The only answer to the question 'What does a word really mean?' is 'To whom?'
- How do we know we understand someone? By giving their words meaning. Our meaning. Not  their meaning. And there is no guarantee that the two meanings are the same.
- The question we want to explore is, what happens to our thoughts when we clothe them in language, and how faithfully are they preserved when our listeners undress them.
- NLP (Neuro Linguistic Programming) has a very useful map of how language operates called the 'Meta Model'. The Meta Model uses language to clarify language, preventing you from deluding yourself that you understand what words mean; it reconnects language with experience.
- The Meta Model is a series of questions that seek to reverse and unravel the deletions and distortions and generalizations of language. These questions aim to fill in the missing information, reshape the structure and elicit specific information to make sense of the communication.
- Mental map of the speaker
- In an everyday context, the Meta Model gives you a systematic way of gathering information, when you need to know more precisely what a person means. It is a skill that is well worth learning.

Filling Knowledge Gaps in Text for Machine Reading
- In this work they fill the missing gaps with the help of external knowledge bases
- However in our work we suggest that filling the missing information might be hard but asking a question that would point to it is easier and hence more helpful.

Bootstrapping semantic parsing from conversations
- System asks clarification question when asked to parse a sentence 
- Their work is at the sentence level; ours is at discourse level. Also our work is for asking questions about missing information instead of clarification


\section{Dataset creation}


- Posts, Post history with edit history
- extract comments on the post that have question marks
- we match the edit made to the post with the question comment 


\section{Task description}



\section{Experiments and Results}\label{experiments_results}

\section{Analysis}

\section{Conclusion}

In the ongoing effort of making computers better at understanding information, if one were to find something missing, the appropriate question to be asked would be \textit{``How can we make computers better at asking questions? "}

\fi
\bibliography{question_generation}
\bibliographystyle{acl2016}


\end{document}
